\documentclass[twoside]{article}

\usepackage{amsfonts,amssymb,amsbsy,textcomp,marvosym,amsmath,caption,threeparttable,amsthm,subfigure}
\usepackage{eurosym,mathrsfs,fancyhdr,CJK,multicol,graphics,indentfirst,color,bm,upgreek,booktabs,graphicx}
\usepackage{amsmath,amsfonts,amssymb}
\usepackage{amsthm}
\usepackage{setspace}
\usepackage{algpseudocode}
\usepackage[linesnumbered,boxed,ruled,commentsnumbered]{algorithm2e}
\usepackage{graphicx}
\usepackage{xspace}
\looseness=-1
%------------Page layout and margin and Headrule-------------
\headsep=5mm \headheight=4mm \topmargin=0cm \oddsidemargin=-0.5cm
\evensidemargin=-0.5cm \marginparwidth=0pt \marginparsep= 0pt
\marginparpush=0pt \textheight=23.1cm \textwidth=17.5cm \footskip=8mm
\columnsep=7mm \setlength{\doublerulesep}{0.1pt}
\renewcommand{\thefootnote}{\fnsymbol{footnote}}
\footnotesep=3.5mm\arraycolsep=2pt
\font\tenrm=cmr10
%===========================================================
\def\footnoterule{\kern 1mm \hrule width 10cm \kern 2mm}
\def\rmd{{\rm d}} \def\rmi{{\rm i}} \def\rme{{\rm e}}
\def\sj#1{$^{[#1]}$}\def\lt{\left}\def\rt{\right}
\renewcommand{\captionfont}{\footnotesize}
\renewcommand\tablename{\bf \footnotesize Table}
\renewcommand\figurename{\footnotesize Fig.\!\!}
\captionsetup{labelsep=period}%
\allowdisplaybreaks
\sloppy
\renewcommand{\headrulewidth}{0pt}
\catcode`@=11
\def\title#1{\vspace{3mm}\begin{flushleft}\vglue-.1cm\Large\bf\boldmath\protect\baselineskip=18pt plus.2pt minus.1pt #1
\end{flushleft}\vspace{1mm} }
\def\author#1{\begin{flushleft}\normalsize #1\end{flushleft}\vspace*{-4pt} \vspace{3mm}}
\def\address#1#2{\begin{flushleft}\vglue-.35cm${}^{#1}$\small\it #2\vglue-.35cm\end{flushleft}\vspace{-2mm}\par}
\def\jz#1#2{{$^{\footnotesize\textcircled{\tiny #1}}$\footnotetext{$^{\footnotesize\textcircled{\tiny #1}}$#2}}}
\def\jzd#1#2{$^{\footnotesize\textcircled{\tiny{#1}}}$\footnotetext{$^{\footnotesize\textcircled{\tiny{#1}}}$#2}}
\catcode`@=11
\def\section{\@startsection{section}{1}{\z@}%
 %{-3.5ex \@plus -1ex \@minus -.2ex}%
 {-3ex \@plus -.3ex \@minus -.2ex}%
 {2.2ex \@plus.2ex}%
{\normalfont\normalsize\protect\baselineskip=14.5pt plus.2pt minus.2pt\bfseries}}
\def\subsection{\@startsection{subsection}{2}{\z@}%
 %{-3.25ex\@plus -1ex \@minus -.2ex}%
 {-3ex\@plus -.2ex \@minus -.2ex}%
 {2ex \@plus.2ex}%
{\normalfont\normalsize\protect\baselineskip=12.5pt plus.2pt minus.2pt\bfseries}}
\def\subsubsection{\@startsection{subsubsection}{3}{\z@}%
 %{-3.25ex\@plus -1ex \@minus -.2ex}%
 {-2.2ex\@plus -.21ex \@minus -.2ex}%
 {1.4ex \@plus.2ex}
{\normalfont\normalsize\protect\baselineskip=12pt plus.2pt minus.2pt\sl}}
\def\proofname{{\indent \it Proof.}}
%===========================================================ÒÔÉϲ»¶¯

\newtheorem{theorem}{Theorem}

\newcommand\ltl{\textsf{LTL}\xspace}
\newcommand\ltlf{\textsf{LTL}$_f$\xspace}
\renewcommand{\tt}{{\sf tt}\xspace}
\newcommand{\ff}{{\sf ff}\xspace}
\newcommand{\X}{\mathcal{X}}
\newcommand{\N}{\mathcal{N}}
\newcommand{\U}{\mathcal{U}}
\newcommand{\R}{\mathcal{R}}
\newcommand{\G}{\Box}
\newcommand{\F}{\Diamond}
\newcommand{\cf}{{\sf CF}\xspace}
\newcommand{\A}{\mathcal{A}}

\pagestyle{fancy}
\fancyhf{}% Çå¿Õҳüҳ½Å
\fancyhead[LO]{\small\sl Shortened Title Within 45 Characters}%
\fancyhead[RO]{\small\thepage}
\fancyhead[LE]{\small\thepage}
\fancyhead[RE]{\small\sl J. Comput. Sci. \& Technol.}
\setcounter{page}{1}
\begin{document}
\thispagestyle{empty}
\vspace*{-13mm}
\noindent {\small Yingying Shi, Shengping Xiao, Jianwen Li, Geguang Pu. SAT-Based Automata Construction for LTL over Finite Traces. JOURNAL OF COMPUTER SCIENCE AND TECHNOLOGY}
%===========================================================
\vspace*{2mm}

\title{SAT-Based Automata Construction for LTL over Finite Traces} 
\author{Yingying Shi, Shengping Xiao, Jianwen Li, and Geguang Pu}
\textit{Addresss:Software Engineering Institute, East China Normal University, Shanghai 200062, China  }             \\                           
Email: 51184501042@stu.ecnu.edu.cn 

%\footnotetext{{}\\[-4mm]\indent\quad Regular Paper}

\noindent {\small\bf Abstract} \quad  {\small{In this paper,we consider Linear Temporal Logic over finite traces. We denote this logic by ($LTL_f$). We propose here a novel approach to translating an $LTL_f$ formula $\phi$ to a Non-deterministic Finite Automaton (NFA) $A_{\phi}$. We first translate the given $LTL_f$ formula $\phi$ into Next Normal Form (XNF). Then we use a SAT-based framework to construct the corresponding NFA. Notably, the construction we present here is an on-the-fly construction.}}

\vspace*{3mm}

\noindent{\small\bf Keywords} \quad {\small $LTL_f$, SAT-based, on-the-fly construction}

\vspace*{4mm}

\baselineskip=18pt plus.2pt minus.2pt
\parskip=0pt plus.2pt minus0.2pt
\begin{multicols}{2}

\section{Introduction}

Linear Temporal Logic (\ltl) was first introduced into Computer Science in 1977~\cite{Pnu77}, since when it has been widely used in formal verification like model checking ~\cite{CGD99}, runtime verification~\cite{BLS11} and property synthesis~\cite{JGWB07,BFJ12}. Beyond that, \ltl also receives a lot of interests from other research communities such as software engineering~\cite{BKMR15} and artificial intelligence~\cite{BK98}. The standard \ltl formulas are interpreted over infinite (linear) traces, which are suitable for describing non-terminating systems' infinite behaviors. Meanwhile, AI applications like motion planning \cite{BK98,DV99,CDV02,PLGG11,CBMM17}, plan constraints \cite{BK00,Gab04} and user preferences \cite{BFM06,BFM11,SBM11}, make more concern on the systems' finite behaviors. 

Recently, \ltlf, a variant of \ltl whose semantics are interpreted over finite traces, has raised great interests from the AI community \cite{GV13,GV15}. Researches on \ltlf satisfiability checking \cite{LZPVH14,LRPZV19} and \ltlf synthesis \cite{GV15} have been extensively investigated. To solve these problems, the crux is to translate the \ltlf formula to its equivalent finite automaton. Constructing the \NFA is sufficient for the satisfiability checking, while constructing the \DFA is required for \ltlf synthesis. This paper focus on the \NFA and \DFA generation from a given \ltlf formula. 

It is well-known that every \ltl formula, which is interpreted over infinite semantics, has a equivalent B\"uchi automata such that they accept the same languages \cite{GPVW95}. 







\section{Preliminaries}
This section introduces the basic concepts w.r.t the linear temporal logic over finite traces (\ltlf) and finite automata. 
\subsection{LTL over finite Traces (\ltlf)}
Classical \ltl formulas are interpreted over infinite traces, which are widely used as the formal property languages to describe system behaviors. Meanwhile, the formulas of \ltlf, which is a variant of \ltl, are interpreted over finite traces. \ltlf are so far widely used in the AI community to describe finite system bahaviors. Given a set $P$ of atomic propositions, the syntax of an \ltlf formula $\phi$ is shown as below.

\begin{center}
$\phi:=\tt\ |\ \ff\ |\ p\ |\neg\phi\ |\ \phi \vee \phi\ |\ \phi \wedge \phi\ |\ \X \phi\ |\ \N \phi\ |\ \phi \U \phi\ |\ \phi \R \phi$ 
\end{center}

In the above, $p \in P$ is an \emph{atom}, while $\X$ (strong Next), $\N \phi$ (weak Next), $\U$ (Until), and $\R$ (Release) are temporal operators. We say $\phi$ is a \emph{literal} if $\phi$ is a proposition or the negation of a proposition. $\X$ and $\N$ (resp. $\U$ and $\R$) are dual operators, i.e. $\X\phi\equiv \neg(\N\neg\phi)$ (resp. $\phi_1\U\phi_2 \equiv \neg (\neg\phi_1 \R\neg\phi_2)$) is semantically true. Boolean operators such as $\rightarrow$ and $\leftrightarrow$ can be represented by the combination $(\neg, \vee)$ or $(\neg, \wedge)$, respectively. Furthermore, we denote the constant \textsf{true} as \tt and \textsf{false} as \ff. We use the particular notation $\F$ (Future) (resp. $\G$ (Global)) for $\U$ (resp. $\R$) such that $\F\phi \equiv \tt\U\phi$ (resp. $\G\phi \equiv \ff \R\phi$) is semantically true. Let $\Sigma = 2^P$ be the set of alphabet and $\eta\in\Sigma^+$ is a finite trace defined over $\Sigma$. We use $\eta[i]$ to denote the $i$-th element of $\eta$, use $\eta_i$ to denote the suffix of $\eta$ starting from position $i$ (including $i$), and use $|\eta|$ to denote the length of $\eta$. Then the semantics of \ltlf formulas are interpreted as follows.
\begin{itemize}
\item  $\eta \models \tt$ and $\eta \not\models \ff$;
\item  If $\phi = p$ is a literal, $\eta \models \phi$ iff $p \in \eta[0]$;
\item  If $\phi = \X \psi$, $\eta \models \phi$ iff $|\eta|>1$ and $\eta_{1} \models \psi$;
\item  If $\phi = \N \psi$, $\eta \models \phi$ iff either $|\eta| = 1$, or $\eta_{1} \models \psi$;
\item  If $\phi = \phi_{1} \U \phi_{2}$, $\eta \models\phi$ iff there exists $0 \leq i<|\eta|$ such that $\eta_{i} \models \phi_{2}$, and for every $0 \leq j<i$ it holds that $\eta_{j} \models \phi_{1}$;
\item  If $\phi = \phi_{1} \R \phi_{2}$, $\eta \models \phi$ iff either for every $0 \leq i<|\eta|$ it holds $\eta_{i}\models \phi_{2}$, or there exists $0 \leq i<|\eta|$ such that $\eta_{i} \models \phi_{1}$ and for all $0 \leq j \leq i$ it holds $\eta_{j} \models \phi_{2}$;
\item  If $\phi = \phi_{1} \wedge \phi_{2}$, $\eta \models \phi$ iff $\eta \models \phi_{1}$ and $\eta \models \phi_{2}$;
\item  If $\phi=\phi_{1} \vee \phi_{2}$, $\eta \models \phi$ iff $\eta \models \phi_{1}$ or $\eta \models \phi_{2}$.
\end{itemize}

We use $\cl(\phi)$ to denote the set of subformulas of $\phi$ \cite{}, whose formal definition are as follows: (1) $\phi\in\cl(\phi)$; (2) if $op (\psi)\in\cl(\psi)$, so does $\psi\in\cl(\psi)$, where $op$ is $\X,\N,\neg,\F$ or $\G$; (3) if $\psi_1\ op\ \psi_2\in\cl(\phi)$, so does $\psi_1\in\cl(\phi)$ and $\psi_2\in\cl(\phi)$, where $op$ is $\U,\R,\wedge$ or $\vee$. In the rest of the paper, all \ltlf formulas are considered in the \emph{Negated Normal Form} (\NNF), i.e., every negation appears only in front of an atom. It is not hard to see that every \ltlf formula can be converted to the equivalent \NNF, and the conversion cost is linear to the size of the formula.

\subsection{Finite Automata}
\noindent A finite automaton $\A$ is a tuple $(\Sigma, S, \rho, s_0, \Omega)$ where 
\begin{itemize}
	\item $\Sigma$ is the set of alphabet;
	\item $S$ is the set of states;
	\item $\rho: S\times\Sigma \rightarrow 2^S$ is the transition function. We say $(s_1,\omega, s_2)$ is a \emph{transition} iff $s_2\in\rho(s_1,\omega)$;
	\item $s_0\in S$ is the initial state;
	\item $\Omega$ is the accepting conditions. We call $\A$ is a state-based (resp. transition-based) automaton if $\Omega\subseteq S$ (resp. $\Omega\subseteq \rho$) is the set of accepting states (resp. accepting transitions).
\end{itemize}
Let $\eta=\omega_0\omega_1\ldots\in\Sigma^+$ be a finite trace over $\Sigma$. A run $r$ of $\A$ on $\eta$ is a finite state sequence $s_0 s_1\ldots$ such that $s_0$ is the initial state and $s_{i+1}\in\rho(s_i,\omega_i)$ for $0\leq i < |\eta|$. We say $\eta$ is accepted by $\A$ iff there is a run $r$ of $\A$ on $\eta$ such that $r$ ends with some accepting condition in $\Omega$. $\A$ is called deterministic iff $|\rho(s,\omega)| \leq 1$ for every $s\in S$ and $\omega\in\Sigma$; otherwise, $\A$ is nondeterministic. 

We consider two different finite automata, i.e., the state-based and transition-based, which vary according to the accepting conditions. In the rest of the paper, a finite automaton is state-based if it is not stated explicitly.  

\begin{theorem}[\cite{GV13}]
Given an \ltlf formula $\phi$, there is a nondeterministic finite automaton $\A$ such that $\eta\models\phi$ iff $\eta$ is accepted by $\A$, for a finite trace $\eta\in\Sigma^+$.
\end{theorem}
\iffalse
\begin{proof} 
($\Leftarrow$) If $\eta$ can be accepted by the \NFA $\A_{\phi}$, there exists a finite trace $\eta$ s.t. the corresponding run $r$ of $\A_{\phi}$ on $\eta$ ends with an accepting state $\phi_n$. According to Definition \ref{def:ltlf2nfa}, $\eta \models \phi$. 

($\Rightarrow$) Since $\eta$ is a finite trace, $\eta \models \phi$. According to Lemma \ref{lem:finiteSat}, there is a run $r= \phi_0 \to \phi_1 \to ...\phi_n$ such that $\phi_n$ is an accepting state. According to Definition \ref{def:ltlf2nfa}, $\eta$ can be accepted by the \NFA $\A_{\phi}$.
\end{proof}
\fi


\section{SAT-Based Automata Construction}
In this section, we construct the $LTL_f$ transition system by SAT solvers. Before defining the transition system, we need to introduce some definitions.\\
\subsection{$LTL_f$ Transition System}
First, we need to consider $LTL_f$ formulas as propositional formulas. And to this end, all temporal subformulas of $\phi$ are treated as atomic propositions. \\
{\bf Definition 1} (Propositional Atoms). {\it For an $LTL_f$ formula $\phi$, we define $PA(\phi)$ as follows:\\
1) If $\phi$ is an atom, Next, Until or Release formula, $PA(\phi)=\{\phi\}$;\\
2) If $\phi = (\neg \psi)$, then $PA(\phi)= PA(\psi)$;\\
3) If $\phi = (\phi_1 \wedge \phi_2)$ or $(\phi_1 \vee \phi_2)$, then $PA(\phi) = PA(\phi_1) \cup PA(\phi_2)$.
}     \\
Consider $\phi$ = $(a \wedge ((\neg c \wedge a)\U b) \wedge (c \wedge \X (a \vee b)))$. In this case, we have $PA(\phi)=\{a, c, (\neg c \wedge a)\U b), (\X (a \vee b))\}$. 
{\bf Definition 2} (Propositional Formulas). {\it Given an $LTL_f$ formula $\phi$, let $\phi^{p}$ be a propositional formula over $PA(\phi)$. \\}
Consider  $\phi$ = $(a \wedge ((\neg c \wedge a)\U b) \wedge (c \wedge \X (a \vee b)))$. From definition 2, the corresponding propositional formula $\phi^p = (a \wedge p_1) \wedge (c \wedge p_2)$, in which $p_1, p_2$ are two boolean variables denoting the truth values of $((\neg c \wedge a)\U b)$ and $(c \wedge \X (a \vee b))$. \\
{\bf Definition 3} (Next Normal Form). {\it For an $LTL_f$ formula $\phi$, $\phi$ is in Next Normal Form(XNF) if every subformula $\X \phi$ is of the form $\X \X \phi^{\prime}, \X a$ or $\X \neg a$ for some $a \in \Sigma$. The Next Normal Form $LTL_f$ formula $\phi$ is defined as follows:\\ 
1) If $\phi$ is a literal l, $xnf(\phi) = \phi \wedge \X(tt)$;\\
2)  $xnf(tt) =  tt$, $xnf(ff) =  \varnothing$;\\
3) If $\phi = \X(\psi)$, $xnf(\phi) = tt \wedge \X(\psi) $;\\
4) If $\phi = (\phi_1 \wedge \phi_2)$, $xnf(\phi) = \{\alpha_1 \wedge \alpha_2)\wedge \X(\psi_1 \wedge \psi_2)| \forall i = 1,2.$ $\alpha_i \wedge \X(\psi_i)\in xnf(\phi_i)$; \\
5) If $\phi = (\phi_1 \vee \phi_2), xnf(\phi) = xnf(\phi_1)\vee xnf(\phi_2)$; \\
6) If $\phi = (\phi_1 \U \phi_2), xnf(\phi) = xnf(\phi_2) \vee (xnf(\phi_1) \wedge \X(\phi))$; \\
7) If $\phi = (\phi_1 \R \phi_2), xnf(\phi) = xnf(\phi)\vee (xnf(\phi_2) \wedge \X(\phi))$; \\
}    
By restricting $LTL_f$ formulas $\phi$ to XNF, we can obtain a satisfying assignment of $\phi^{p}$, leveraging a SAT solver.\\
{\bf Definition 4} (Accepting States). {\it Given an $LTL_f$ formula $\phi$. Suppose that $s_1$ is a state of a transition system $T_{\phi}$. We say  $s_1$ is an accepting state iff there exists a transition $ s_1 \overset{\alpha}{\rightarrow}s_1$, and moreover $\alpha \models s_1$. }   \\   
{\bf Definition 5} (Non-deterministic Finite Automaton). {\it An non-deterministic finite automaton is referred to by its acronym: NFA. Given an $LTL_f$ formula $\phi$ and its literal set L, $\Sigma = 2^{L}$. The representation of an NFA is a listing of five components: $A = (s_0,S, T, \Sigma, F)$ where:
1) $s_0 = \{\phi \}$ is the initial state.\\
2) S is a finite set of states.\\
3) $T:  S \times \Sigma \to 2^S$ is the transition function, where $s_2 \in T(s_1, \alpha) (\alpha \in \Sigma)$ holds iff $ \alpha \wedge \X(s_2) \in xnf(s_1)$.\\
4) $\Sigma$ is a finite set of input symbols.\\
5) F, a subset of S, is the set of accepting states.\\ }
Intuitively, $T(s, a)$ is the set of states that A can move into when it is in state s and it reads the symbol a.  \\
A run r of A on a finite word $w = a_0, ..., a_{n-1} \in \Sigma^{*}$ is a sequence $s_0, ..., s_n $of n+1 states in S such that $s_0 \in S_0$, and $s_{i+1} \in T(s_i, a_i)$ for $0 \leq i < n$. Notably, a nondeterministic automaton can have many runs on a given input word. A run r is accepting if $s_n \in F$.   \\
{\bf Lemma 1.}  {\it For a finite trace $\eta \in \Sigma^{*}$ and $LTL_f$ formula $\phi$, if $\eta = w_0$, then $\eta \models \phi$ holds iff: \\ 
$\bullet$ $\eta \models tt$ and $\eta \not \models $ ff;\\
$\bullet$ If $\phi=p$ is a literal, then return true if $p \in \eta$, otherwise return false;\\
$\bullet$ If $\phi=\X \psi$, then return false;\\
$\bullet$ If $\phi=\N \psi,$ then return true;\\
$\bullet$ If $\phi=\phi_{1} \U \phi_{2}$ or  $\phi=\phi_{1} \R \phi_{2}$, then return  $\eta \models \phi_{2}$; \\
$\bullet$ If $\phi=\phi_{1} \wedge \phi_{2},$ then return $\eta \models \phi_{1}$ and $\eta \models \phi_{2}$; \\
$\bullet$ If $\phi=\phi_{1} \vee \phi_{2},$ then return $\eta \models \phi_{1}$ or $\eta \models \phi_{2}$.}

\begin{proof}
The lemma is self-evident which can be derived from the semantics of $LTL_f$.
\end{proof}

Now we characterize the satisfaction relation for finite sequences: \\
{\bf Lemma 2.}  {\it For a finite trace $\eta = w_0w_1...w_n \in \Sigma^{*}$ and $LTL_f$ formula $\phi$. $\eta \models \phi$ iff there exists a run $r = \phi \overset{\alpha_0}{\rightarrow}\phi_1\overset{\alpha_1}{\rightarrow}\phi_2...\overset{\alpha_n}{\rightarrow}\phi_{n+1}$ in NFA $A_{\phi}$ s.t. every $0 \leq i \leq n$ it holds that $w_i \models \alpha_i$ and $\eta_i \models \phi_i $.}
\begin{proof}
The Lemma above can be proved by recursively applying the first item if $n = 0$, and the second item if $n \geq 1$. \\
1) If n = 0, then $\eta \models \phi$ iff there exists $ \alpha \wedge \X\phi \in xnf(\phi)$ or $\alpha_i \wedge \X(tt) \in xnf(\phi)$ s.t. $w_0 \models \alpha, \alpha \models \phi$.\\
2) If $n \geq 1$, then  $\eta \models \phi$ iff there exists  $ \alpha_i \wedge \X\phi_i \in xnf(\phi)$ s.t. $w_0 \models \alpha_i$ and $\eta_1 \models \phi_i $.

1) $\bullet$ If $\phi = p$, then $\eta \models \phi$ iff $w_0 \models$ p, and p $\models \phi$ hold. According to Definition 2, $p \wedge \X(tt) \in xnf(p)$.\\
$\bullet$ If $\phi=\X \psi$, according to Lemma 1, $\eta \models \X \psi$ is always false. Similiarly, If $\phi=\N \psi$, $\eta \models \N \psi$ is always true.\\
$\bullet$ If $\phi=\phi_{1} \U \phi_{2}$, according to Lemma 1, $\eta \models \phi$ iff $\eta \models \phi_{2}$. $xnf(\phi_{2})=\phi_{2} \wedge \X(tt)$, then  $w_0 \models \phi_{2}$, and $\phi_{2} \models \phi_{2}$, thus the proof is done. The proof for the case $\phi=\phi_{1} \R \phi_{2}$ is similar. \\
$\bullet$ If $\phi=\phi_{1} \wedge \phi_{2}$, according to Definition 3, there exists $\beta_i \wedge \X\psi_i \in xnf(\phi)(i = 1,2).$ Let $\alpha = \beta_1 \wedge \beta_2$ and $\psi = \phi_1 \wedge \phi_2$, then $w_0 \models \alpha$ and $\alpha \models \phi$ hold. Similarly, the proof for the case when $\phi = \phi_1 \vee \phi_2$ is the same. \\
2) The second item can be proved by structural induction over $\phi$.\\
 $\bullet$ If $\phi_i = p$, then $\eta \models \phi$ iff $w_0 \models \phi$, and $\eta_1 \models tt$, according to Definition 2, $xnf(p)=p \wedge \X(tt)$, then proof is done.\\
 $\bullet$ If $\phi_i=\X \psi$ or $\eta \models \N \psi$, then $\eta \models \phi$ iff $w_0 \models tt$ and $\eta_1 \models \psi$. According to Lemma 1, $tt \wedge \X\phi_2 \in xnf(\phi)$.\\
$\bullet$ If $\phi=\phi_1 \wedge \phi_2$, then $\eta \models \phi_1 \U \phi_2$ iff $\eta \models \phi_1$ and $\eta \models \phi_2$, and iff there exists $\beta_i \wedge \X\psi_i \in xnf(\phi_i)(i=1,2)$ s.t. $w_0 \models \beta_1 \wedge \beta_2$ and $\eta_1 \models \psi_1 and \psi_2$. According to Definition ,$(\beta_1 \wedge \beta_2)\wedge \X(\psi_1 and \psi_2)$. The proof for the case when $\phi_i=\psi_{1} \vee \psi_{2}$ is similar. \\
 $\bullet$ If $\phi_i=\phi_{1} \U \phi_{2}$, then $\eta \models \phi$ iff $\eta \models \phi_2$ or $\eta \models (\phi_1 \wedge \X\phi)$. If $\eta \models \phi_2$ holds, then $\eta \models \phi$ iff there exists $\alpha_i \wedge \X\phi_i \in xnf(\phi_2)$ s.t. $w_0 \models \alpha_i$ and $\eta_1 \models \phi_i$. According to Definition , there exists $\alpha_i \wedge \X\phi_i \in xnf(\phi_2)$. On the other hand, if $\eta \models (\phi_1 \wedge \X\phi)$ holds, the proof for the case when the operator is $\wedge$ is done. So $\eta \models \phi$ iff there exists $\alpha_i \wedge \phi_i \in xnf(\phi_2)$ s.t. $w_0 \models \alpha_i$ and $\eta_1 \models \phi_i$. The proof the case when $\phi=\phi_1 \R \phi_2$ is similar.
\end{proof}
Lemma 2 demotes that, to check whether a finite trace $\eta = w_0 w_1 ... w_n$ satisfies an $LTL_f$ formula $\phi$, iff we can find a trace of an NFA $A_\phi$ s.t. $\eta$ can finally reach an accepting state. Now we show the main theorem of this paper.\\
{\bf Theorem 1.}  {\it For a finite trace  $\eta = w_0w_1...w_n $ and $LTL_f$ formula $\phi$, $\eta \models \phi$ iff $\eta$ can be accepted by the NFA $A_\phi$.}
\begin{proof} $(\Leftarrow)$ If $\eta$ can be accepted by the NFA $A_{\phi}$, then there exists a finite trace $\eta$ s.t. the corresponding run r of NFA $A_{\phi}$ on $\eta$ ends with  an accepting state $\phi_n$. According to Lemma 2, $\eta \models \phi$. \\
$(\Rightarrow)$ Since $\eta$ is a finite trace, $\eta \models \phi$. According to Lemma 2, there is a run r= $\phi_0 \to \phi_1 \to ...\phi_n$ where $\phi_n$ is an accepting state. Then $\eta \models \phi$(according to Definition 6).
\end{proof}


\section{DFA}
We define the propositional assignments set of a $LTL_f$ formula $\phi$ as $SA(\phi)=\{A|\ A\ is\ a\ propositional\ assignment\ of\ \phi^p\}$. And we denote $tran(SA)=\{L(A)|\ A\in SA\}$ and $d(\sigma,SA)=\{X(A)|A\in SA\land\sigma=L(A)\}$.

Given an $LTL_f$ formula $\phi$ and its literal set $L$, We define the automaton $A_{\phi}=(S,\Sigma,s_0,T,F)$ for:\\
$S\subseteq2^{2^{cl\phi} }$\\
$\Sigma=2^L$\\
$s_0=\{\{\phi\}\}$\\
$T: S\times\Sigma\to~S$, $s_1\times\sigma\to s_2\in T$ iff $\sigma \in tran(\phi)\land s_2=d(\sigma,SA(\phi))$, where $\phi=(xnf(\mathop{\lor}\limits_{x_i\in s_1}(\mathop{\land} x_i)))^p$\\
$F: s\in F$ iff there exists a transition $s_1\times\sigma\to s$ $s.t.$~$\land\sigma_i\models \mathop{\lor}\limits_{x_i\in s}(\mathop{\land} x_i)$







\section{ Conclusions}

Although a conclusion may review the main points of the paper, do not replicate the abstract as the conclusion. A conclusion might elaborate on the importance and results of the work, and/or suggest applications and extensions.

\vspace{2mm}

 

\begin{thebibliography}{99}
\footnotesize
\itemsep=-3pt plus.2pt minus.2pt
\baselineskip=13pt plus.2pt minus.2pt
\bibitem{1}Sayah J Y, Kime C R. Test scheduling in high performance VLSI system implementations. {\it IEEE Trans. Computers}, 1992, 41(1): 52-67.  [\textcolor{blue}{example for journal paper}]


\end{thebibliography}

\label{last-page}
\end{multicols}
\end{document}

