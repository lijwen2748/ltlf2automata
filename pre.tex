\section{Preliminaries}
\subsection{LTL over finite Traces (\ltlf)}
Classical \ltl formulas are interpreted over infinite traces, which are widely used as the formal property languages to describe system behaviors. Meanwhile, the formulas of \ltlf, which is a variant of \ltl, are interpreted over finite traces. \ltlf are so far widely used in the AI community to describe finite system bahaviors. Given a set $P$ of atomic propositions, the syntax of an \ltlf formula $\phi$ is shown as below.

\begin{center}
$\phi:=\tt\ |\ \ff\ |\ p\ |\neg\phi\ |\ \phi \vee \phi\ |\ \phi \wedge \phi\ |\ \X \phi\ |\ \N \phi\ |\ \phi \U \phi\ |\ \phi \R \phi$ 
\end{center}

In the above, $p \in P$ is an \emph{atom}, while $\X$ (strong Next), $\N \phi$ (weak Next), $\U$ (Until), and $\R$ (Release) are temporal operators. We say $\phi$ is a \emph{literal} if $\phi$ is a proposition or the negation of a proposition. $\X$ and $\N$ (resp. $\U$ and $\R$) are dual operators, i.e. $\X\phi\equiv \neg(\N\neg\phi)$ (resp. $\phi_1\U\phi_2 \equiv \neg (\neg\phi_1 \R\neg\phi_2)$) is semantically true. Boolean operators such as $\rightarrow$ and $\leftrightarrow$ can be represented by the combination $(\neg, \vee)$ or $(\neg, \wedge)$, respectively. Furthermore, we denote the constant \textsf{true} as \tt and \textsf{false} as \ff. We use the particular notation $\F$ (Future) (resp. $\G$ (Global)) for $\U$ (resp. $\R$) such that $\F\phi \equiv \tt\U\phi$ (resp. $\G\phi \equiv \ff \R\phi$) is semantically true. Moreover, we use $\cf(\phi)$ to denote the set of conjuncts in $\phi$, i.e., $\cf{\phi} = \{\phi_i | 1\leq i\leq n\}$ for $\phi = \bigwedge_{1\leq i\leq n}\phi_i$. Notably, the root of these conjuncts is not a conjunction. Let $\Sigma = 2^P$ be the set of alphabet and $\eta\in\Sigma^+$ is a finite trace defined over $\Sigma$. We use $\eta[i]$ to denote the $i$-th element of $\eta$, use $\eta_i$ to denote the suffix of $\eta$ starting from position $i$ (including $i$), and use $|\eta|$ to denote the length of $\eta$. Then the semantics of \ltlf formulas are interpreted as follows.
\begin{itemize}
\item  $\eta \models \tt$ and $\eta \not\models \ff$;
\item  If $\phi = p$ is a literal, $\eta \models \phi$ iff $p \in \eta[0]$;
\item  If $\phi = \X \psi$, $\eta \models \phi$ iff $|\eta|>1$ and $\eta_{1} \models \psi$;
\item  If $\phi = \N \psi$, $\eta \models \phi$ iff either $|\eta| = 1$, or $\eta_{1} \models \psi$;
\item  If $\phi = \phi_{1} \U \phi_{2}$, $\eta \models\phi$ iff there exists $0 \leq i<|\eta|$ such that $\eta_{i} \models \phi_{2}$, and for every $0 \leq j<i$ it holds that $\eta_{j} \models \phi_{1}$;
\item  If $\phi = \phi_{1} \R \phi_{2}$, $\eta \models \phi$ iff either for every $0 \leq i<|\eta|$ it holds $\eta_{i}\models \phi_{2}$, or there exists $0 \leq i<|\eta|$ such that $\eta_{i} \models \phi_{1}$ and for all $0 \leq j \leq i$ it holds $\eta_{j} \models \phi_{2}$;
\item  If $\phi = \phi_{1} \wedge \phi_{2}$, $\eta \models \phi$ iff $\eta \models \phi_{1}$ and $\eta \models \phi_{2}$;
\item  If $\phi=\phi_{1} \vee \phi_{2}$, $\eta \models \phi$ iff $\eta \models \phi_{1}$ or $\eta \models \phi_{2}$.
\end{itemize}

In the rest of the paper, all \ltlf formulas are considered in the \emph{Negated Normal Form} (NNF), i.e., every negation appears only in front of an atom. 

\subsection{Finite Automata}
\noindent A finite automaton $\A$ is a tuple $(\Sigma, S, \rho, s_0, F)$ where 
\begin{itemize}
	\item $\Sigma$ is the set of alphabet;
	\item $S$ is the set of states;
	\item $\rho: S\times\Sigma \rightarrow 2^S$ is the transition function. We say $(s_1,\omega, s_2)$ is a \emph{transition} iff $s_2\in\rho(s_1,\omega)$;
	\item $s_0\in S$ is the initial state;
	\item $F\subseteq S$ is the set of accepting states. 
\end{itemize}
Let $\eta=\omega_0\omega_1\ldots\in\Sigma^+$ be a finite trace over $\Sigma$. A run $r$ of $\A$ on $\eta$ is a finite state sequence $s_0 s_1\ldots$ such that $s_0$ is the initial state and $s_{i+1}\in\rho(s_i,\omega_i)$ for $0\leq i < |\eta|$. We say $\eta$ is accepted by $\A$ iff there is a run $r$ of $\A$ on $\eta$ which ends with some accepting state in $F$. $\A$ is called deterministic iff $|\rho(s,\omega)| \leq 1$ for every $s\in S$ and $\omega\in\Sigma$; otherwise, $\A$ is nondeterministic.   

\begin{theorem}[\cite{GV13}]
Given an \ltlf formula $\phi$, there is an nondeterministic finite automaton $\A$ such that $\eta\models\phi$ iff $\eta$ is accepted by $\A$, for a finite trace $\eta\in\Sigma^+$.
\end{theorem}
\iffalse
\begin{proof} 
($\Leftarrow$) If $\eta$ can be accepted by the \NFA $\A_{\phi}$, there exists a finite trace $\eta$ s.t. the corresponding run $r$ of $\A_{\phi}$ on $\eta$ ends with an accepting state $\phi_n$. According to Definition \ref{def:ltlf2nfa}, $\eta \models \phi$. 

($\Rightarrow$) Since $\eta$ is a finite trace, $\eta \models \phi$. According to Lemma \ref{lem:finiteSat}, there is a run $r= \phi_0 \to \phi_1 \to ...\phi_n$ such that $\phi_n$ is an accepting state. According to Definition \ref{def:ltlf2nfa}, $\eta$ can be accepted by the \NFA $\A_{\phi}$.
\end{proof}
\fi

