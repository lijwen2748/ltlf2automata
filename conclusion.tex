\section{Discussion and Conclusion}\label{sec:discuss}
There are two ways to convert an \ltlf formula to its corresponding automaton. The first method is to generalize the B\"uchi automata with \spot and then convert it to \NFA. Since the \ltl operator used by \spot is defined over infinite words, given an \ltlf formula, a little tricks need to be taken to rewrite it in a way that encompasses the semantics of \ltlf in \ltl. First, introduce a new atomic proposition, "Tail", which is true at first but will eventually become permanently false. Then adjust all the original \ltl operators so that they must be satisfied in the Tail part of the word. Afterwards, \spot converts the resulting formula into a B\"uchi automaton. Finally, remove the Tail property, and simplify the Buchi automata to \NFA.
The second approach is to convert the input \ltlf formulas to first-order logic, leveraging \mona to get the minimum DFA. The automata are represented by shared, multi-terminal BDDs. Both of the two tools utilize BDD techniques. However, our approach is quite straightforward, it converts the \ltlf formula to the equivalent transition-based \NFA (\TNFA) and transition-based \DFA (\TDFA) directly without intermediate steps. In addition to that, our approach utilizes modern \SAT techniques to construct the automaton on-the-fly. 

In conclusion, our main contribution is that, given an \ltlf formula, it is translated directly into \TNFA and \TDFA without intermediate steps. Apart from that, \SAT techniques are also used to generate the automata on-the-fly. We will leave the comparison with \spot and \mona in our future work.
