\section{Discussion and Conclusion}\label{sec:discuss}
As far as we know, there are two approaches to convert an \ltlf formula to its equivalent finite automaton. The first one is to generate the B\"uchi automaton using \spot and then convert it to the corresponding \NFA. Since \spot only supports the translation of automata from \ltl formulas that are over infinite traces, one should first rewrite the input \ltlf formula to its equi-satisfiable \ltl formula, in which a new proposition \emph{alive} should be introduced to identify the \ltlf-semantic satisfaction under the \ltl semantics. After that, the equivalent B\"uchi automaton w.r.t. the \ltl formula can be obtained via \spot, whose transitions are labeled with transitions containing either \emph{alive} or \emph{$\neg$alive}. Finally, by deleting all transitions with \emph{$\neg$alive} labeled, removing the \emph{alive} proposition on the other transitions and simplifying the left parts, one is able to obtain the finite \NFA. \spot utilizes the \BDD techniques to achieve the automata translation. An illustrative example is shown at the website of \spot\footnote{\url{https://spot.lrde.epita.fr/tut12.html}}.

The second approach converts the input \ltlf formula to the monadic First-Order Logic (FOL) and then leverages the \mona tool \cite{EKM98} to construct the minimum \DFA \cite{ZTLPV17}. The correctness guarantee underlies this approach replies on the fact that there is an equivalent monadic FOL formula for an arbitrary \ltlf formula. The generated automaton is represented by the shared, multi-terminal \BDD. 

Our approach introduced in this paper differs from the above two methods as follows. (1) The translations from an \ltlf formula to its equivalent \TNFA and \TDFA are straightforward without involving any intermediate processes. (2) The resulting automata of our approach are transition-based, which has been shown more efficient than the state-based ones from previous works \cite{GL02}. (3) Most importantly, our approach leverages the \SAT instead of the \BDD techniques for automata construction, which makes it potentially more useful for on-the-fly applications such as planning (via synthesis). Actually, this paper is considered as the pre-processing work for \ltlf on-the-fly synthesis in our future work, which focuses on the task of establishing the fundamental part to construct the automata on the fly. This is also the reason why we did not provide the experimental comparison between our approach and those based on \spot and \mona in terms of the constructed automata in this paper.

To conclude, we present the direct translations from an \ltlf formula to its equivalent \TNFA and \TDFA respectively. The translations leverage the \SAT techniques such that the automata can be constructed on the fly. As far as we know, this is the first work that proposes to generate the transition-based finite automata for the \ltlf formulas, and we will evaluate the corresponding advantages in the real applications such as planning (via synthesis) in our future work. 
