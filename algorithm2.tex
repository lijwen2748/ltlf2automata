\subsection{\ltlf-to-\TDFA}
As shown in Algorithm~\ref{algo:DFAconstruct}, we can construct a \TDFA from an \ltlf formula $\phi$ on-the-fly.

\begin{algorithm}[H]\label{algo:DFAconstruct}
  \caption{Construction of the \TDFA}
  \LinesNumbered
  \KwIn{\ltlf formula $\phi$}
  \KwOut{\TDFA\ ${\A_\phi}^{td}=(\Sigma,Q,\delta,q_0,T^d)$}
  $\Sigma\coloneqq 2^L$\;
  $Q\coloneqq\{q_0\}$\;
  $\delta\coloneqq\emptyset$\;
  $q_0\coloneqq\{\{\phi\}\}$\;
  $T^d\coloneqq\emptyset$\;
  $to\_process\coloneqq\{q_0\}$\;
  \While{$to\_process \neq \emptyset$}
  {
    \textbf{get} a state $q$ from $to\_process$\;
    $\psi\coloneqq\xnf(q)^p$\;
    \textbf{let}  $label\_next$ be a \textbf{map}\;
    \While{$\psi$ is satisfiable}
    {
      \textbf{get} a model $A$ of $\psi$\;
      $label\coloneqq L(A)$\;
      $next\coloneqq X(A)$\;
      \If{$label\_next[label]=NULL$}
      {
        $label\_next[label]\coloneqq \{next\}$\;
      }
      \Else
      {
        $label\_next[label]\coloneqq label\_next[label]\cup\{next\}$\;
      }
      $\psi\coloneqq \psi\land(\neg(\bigwedge A))$\;
    }
    \For{$(Label,Next)$ in $label\_next$}
    {
      \If{$Next\notin Q$}
      {
        \textbf{add} $Next$ into $Q$\;
        \textbf{add} $Next$ into $to\_process$\;
      }
      \textbf{add} $q\times Label\to Next$ into $\delta$\;
      \If{$Label\models q$}
      {
        \textbf{add} $q\times Label\to Next$ into $T^d$\;
      }
    }
    \textbf{remove} $q$ from $to\_process$\;
  }
  \textbf{return} $(\Sigma,Q,\delta,q_0,T^d)$\;
\end{algorithm}
~\\
Obviously, Algorithm~\ref{algo:DFAconstruct} is developed from Definition \ref{def:ltlf2dfa}. In the algorithm, the function of $to\_process$ is the same as in Algorithm~\ref{algo:TNFAconstruct}. Each time the while loop in the first level (Line~7) is executed, a state will be taken from $to\_process$ to solve all its successor states, and new states that have not appeared will be added to $Q$ and $to\_process$. After solving the successor of a state, it will be removed from $to\_process$. When we are to get the successor of a state $q$, we process all models of the $\xnf(q)^p$. The map $label\_next$ is used to integrate the $X(A)$ part of the models with the same $X(A)$ (Line~16-18). Then the for loop (Line~20) adds new states into $Q$ and $to\_process$, and adds the transition in $label\_next$ into $\delta$ and $T^d$ (if accepting condition holds).