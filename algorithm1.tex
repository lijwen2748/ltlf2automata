\section{Implementation}\label{sec:algorithm}
Given an $LTL_f$ formula $\phi$, there is a \NFA $A_{\phi}$ which accepts exactly the same language as $\phi$. To construct the nondeterministic finite automaton (\NFA) for an $ LTL_f$ formula $\phi$, our approach first constructs the corresponding transition-based \NFA, then converts the \TNFA to its equivalent \NFA. The following two subsections will describe this process. The first subsection describes the algorithm of \ltlf-to-\TNFA. The second subsection describes the algorithm of \TNFA-to-\NFA.

\subsection{\ltlf-to-\TNFA}
Given an  $ LTL_f$ formula, We define the construction of the \TNFA  as follows: \\
  \begin{algorithm}[H]
    \SetAlgoNoLine
 \SetKwInOut{Input}{\textbf{Input}}\SetKwInOut{Output}{\textbf{Output}}
 \Input{ An $ LTL_f$ formula $\phi$\\}
    \Output{ TNFA $A^{tn}$\\}
    \BlankLine
 $to\_process:= \{\xnf(\phi)\}$\\
 $xnf(\phi)$ is the initial state in S\\
\textbf{while} $to\_process$ is not empty \textbf{do} \\
\ cur =  $to\_process$.front;  \\
\quad  \textbf{while} true \textbf{do} \\
\qquad   \textbf{if} cur is not satisfiable  \textbf{then} \{ \\
\qquad \quad $to\_process$.pop; \\
\qquad \quad break;     \}\\
 \qquad  next = $\rho(cur, label)$ \\
\qquad\textbf{if} next is not in existing \textbf{then} \{ \\
\qquad \quad $to\_process$.push($next$); \\
\qquad \quad existing.push($next$); \\
\qquad  \quad add the state $next$ to S\\
\qquad  add the transition $cur\tran{label}next$ to T
 \caption{Construction of the \TNFA}
\end{algorithm}

The algorithm constructs the transition system on-the-fly. The initial state is set to be $\{\phi\}$, which is the input formula. $to\_process$ is used to store a collection of states to be processed. Visited states are stored in $existing$. $front$ is used to describe the first element of an array. In a transition $s_1\tran{\omega}s_2$, $\omega$ is the $label$, $s_2$ is the $next$ state. \\
In this algorithm, we first convert the input formula into its Next Normal Form, i.e., $\xnf(\phi)$. Then we use \SAT solvers to check if $\xnf(\phi)$ is satisfiable. If the answer is true, proceed to the next step, and if the answer is false, stop. After that, we leverage $\SAT solvers$ to get a transition $cur\tran{label}next$, and then we check if the state $next$ has been processed. If the answer is true, then we add the transition to ${A_{\phi}}^{tn}$, otherwise, the state $next$ will be added to  $to\_process$,existing and ${A_{\phi}}^{tn}$ respectively.


\subsection{\TNFA-to-\NFA}  
In the previous section, we constructed the state and transition of the \TNFA. This section will focus on the conversion from \TNFA-to-\NFA, with the addition of accepting states. The algorithm is stated as follows:\\

\IncMargin{1em}
 \begin{algorithm}[H]
    \SetAlgoNoLine
 \SetKwInOut{Input}{\textbf{Input}}\SetKwInOut{Output}{\textbf{Output}}
 \Input{ \TNFA $A^{tn}$\\}
    \Output{  \NFA $A^{n}$}
    \BlankLine
 
\qquad  $S' = S\cup \{f\}$   \\
\qquad  \textbf{for} state $s_2$ in S \\
\qquad \quad \textbf{if} $s_2\not = f$ and $s_2\in \rho(s_1, \omega)$\textbf{then} \{  \\
\qquad \qquad  $\rho'(s_1,\omega)$ = $\rho(s_1, \omega)$ \} \\
\qquad \quad \textbf{else} \{  \\
\qquad \qquad add a transition $s_1\tran{\omega}s_3$ in $\rho'$
\} \\
 \caption{\TNFA-to-\NFA}
\end{algorithm}

This algorithm converts the \TNFA to \NFA. S is the set of states in \TNFA, while $S'$ is the set of states in \NFA. f is the only accepting state in \NFA $A^{n}$. \\
