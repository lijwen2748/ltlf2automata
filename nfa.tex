\section{SAT-Based Automata Construction}\label{sec:main}

In this section, we present the construction from an \ltlf formula to the equivalent (T)\NFA and (T)\DFA respectively. 


\subsection{\ltlf-to-(T)\NFA Construction}

To leverage \SAT solvers for the automata construction, we first need to consider \ltlf formulas as propositional formulas. The motivation comes from treating all temporal subformulas of $\phi$ as atomic propositions. 

\begin{definition}[Propositional Atoms]\label{def:pa}
For an \ltlf formula $\phi$, we define the set of propositional atoms, $\PA(\phi)$, recursively as follows.
\begin{itemize}
	\item If $\phi$ is an atom, Next, weak Next, Until or Release formula, $\PA(\phi)=\{\phi\}$;
	\item If $\phi = (\neg \psi)$, $\PA(\phi)= \PA(\psi)$;
	\item If $\phi = (\phi_1 \wedge \phi_2)$ or $(\phi_1 \vee \phi_2)$, $\PA(\phi) = \PA(\phi_1) \cup \PA(\phi_2)$.
\end{itemize}
\end{definition}

Consider the formula $\phi$ = $(a \wedge ((\neg c \wedge a)\U b) \wedge (c \wedge \X (a \vee b)))$ as an example. According to Definition \ref{def:pa}, we have $\PA(\phi)=\{a, c, (\neg c \wedge a)\U b), (\X (a \vee b))\}$. Now we define the propositional formulas over propositional atoms w.r.t. an \ltlf formula.

\begin{definition}[Propositional Formulas]\label{def:pf}
Given an \ltlf formula $\phi$, let $\phi^{p}$ be a propositional formula over $\PA(\phi)$. 
\end{definition}

Consider $\phi$ = $(a \wedge ((\neg c \wedge a)\U b) \wedge (c \wedge \X (a \vee b)))$. From Definition \ref{def:pf}, the corresponding propositional formula $\phi^p = (a \wedge p_1) \wedge (c \wedge p_2)$, in which $p_1, p_2$ are two Boolean variables representing the truth values of $((\neg c \wedge a)\U b)$ and $(c \wedge \X (a \vee b))$. We define the \emph{next normal form} for an \ltlf formula, whose corresponding propositional formula is the actual input to \SAT solvers.

\begin{definition}[neXt Normal Form]\label{def:xnf}
For an \ltlf formula $\phi$, we define its Next Normal Form, i.e., $\xnf(\phi)$, recursively as follows. 
\begin{itemize}
\item $\xnf(\tt) = \tt$ and $\xnf(\ff) = \ff$;
\item If $\phi$ is a literal, $\xnf(\phi) = \phi \wedge \X(\tt)$;
\item If $\phi = \X(\psi)$ or $\phi = \N(\psi)$, $\xnf(\phi) = \X(\psi)$;
\item If $\phi = (\phi_1 \wedge \phi_2)$, $\xnf(\phi) = \xnf(\phi_1)\wedge\xnf(\phi_2)$;
\item If $\phi = (\phi_1 \vee \phi_2), \xnf(\phi) = \xnf(\phi_1)\vee \xnf(\phi_2)$; 
\item If $\phi = (\phi_1 \U \phi_2), \xnf(\phi) = \xnf(\phi_2) \vee (\xnf(\phi_1) \wedge \X(\phi))$; 
\item If $\phi = (\phi_1 \R \phi_2), \xnf(\phi) = \xnf(\phi_2)\wedge (\xnf(\phi_1) \vee \X(\phi))$.
\end{itemize}
\end{definition}

By restricting the \ltlf formula $\phi$ to the \XNF, we can obtain a satisfying assignment of $\phi^{p}$ with the help of the SAT solver.
For a satisfying assignment $A$, we define $X(A)=\{\theta\ |\ \X\theta\in A\}$ and $L(A) =\{l\ |\ l \in A \textit{ is a literal}\}$. Also for the simple description, \textbf{we mix the usage of $X(A)$ to denote the conjunction of all elements in the set, i.e., $\bigwedge_{\psi_i\in X(A)}\psi_i$. The same applies to $L(A)$.}
The following lemma describes the relationship between an \ltlf formula and the satisfying assignment of its corresponding propositional formula.

\begin{lemma}\label{lem:propSat}
Given an \ltlf formula $\phi$ in the \XNF and a finite trace $\eta\in\Sigma^+$ with $|\eta|>1$, $\eta\models\phi$ holds iff there is a satisfying assignment $A$ of $\phi^p$ such that $\eta[0]\models L(A)$ and $\eta_1\models X(A)$.
\end{lemma}
\begin{proof}
$(\Rightarrow)$ The proof can be achieved by induction over the types of $\phi$. Since $\phi$ is in \XNF, $\phi$ cannot be an Until or Release formula.
\begin{itemize}
	\item if $\phi$ is a literal,  we could have a satisfying assignment $A=\{\phi\}$ for $\phi^p$. In this situation, $L(A)=\{\phi\}$ and $X(A)=\{\tt\}$. Therefore, it is true that $\eta[0]\models L(A)$ and $\eta_1\models X(A)$;
	\item if $\phi=\X\psi$, we could have a satisfying assignment $A=\{\X\psi\}$ for $\phi^p$. In this situation, $L(A)=\{\tt\}$ and $X(A)=\{\psi\}$. Considering we have $\eta\models\phi$ i.e., $\eta\models\X\psi$, it is true that $\eta[0]\models L(A)$ and $\eta_1\models X(A)$;
	\item if $\phi = \phi_1\wedge\phi_2$, $\eta\models\phi$ 
implies $\eta\models\phi_1$ and $\eta\models\phi_2$. By assumption hypothesis, there is $A_i$ of $\phi_i^p$ ($i=1,2$) such that 
$\eta[0]\models L(A_i)$ and $\eta_1\models X(A_i)$. Let $A = A_1\cup A_2$, and a consistent $A$, in which either $\psi$ or $\neg \psi$ cannot be together, 
must exists ($A$ may not be unique because $A_1$ and $A_2$ may not be unique). 
Otherwise, there is $\psi\in A_1$ and $\neg\psi\in A_2$ 
such that $\eta$ cannot model $\bigwedge A_1$ and $\bigwedge A_2$ at the same time, which is a contradiction. So $A$ is a satisfying  assignment of $\phi^p$, which leads to $\eta[0]\models L(A)$ and $\eta_1\models X(A)$. 
\item The proof for $\phi=\phi_1\vee\phi_2$ is analogous to the proof when $\phi = \phi_1\wedge\phi_2$.
\end{itemize} 

$(\Leftarrow)$ Let $A$ be a satisfying assignment of $\phi^p$, so $A\models\phi^p$ implies $(\bigwedge A)\Rightarrow \phi$. 
Also, $\eta[0]\models L(A)$ and $\eta_1\models X(A)$ implies $\eta\models \bigwedge A$. Therefore, it is true that $\eta\models \phi$.
\end{proof}

Lemma \ref{lem:propSat} does not cover the case when $|\eta|=1$, which yields the following corollary. 

\begin{corollary}\label{coro:propSat}
Given an \ltlf formula $\phi$ in the \XNF and a finite trace $\eta\in\Sigma^+$ with $|\eta|=1$, $\eta\models\phi$ holds implies there is a satisfying assignment $A$ of $\phi^p$ such that $\eta[0]\models L(A)$.
\end{corollary}
\begin{proof}
The proof is similar to that of Lemma \ref{lem:propSat} for the ($\Rightarrow$) direction, the details of which are omitted here.
\end{proof}
  
Informally speaking, if $\phi$ is a state in the \NFA, $X(\xnf(\phi)^p)$ represents one of its successors and $L(\xnf(\phi)^p)$ represents the label on the transition from $\phi$ to $X(\xnf(\phi)^p)$. Now we present the construction from an \ltlf formula to its equivalent transition-based \NFA. 

   
\begin{definition}[Transition-based NFA (\TNFA)]\label{def:ltlf2nfa} 
Given an \ltlf formula $\phi$, the corresponding transition-based \NFA ${\A_{\phi}}^{tn}$ is defined as a tuple $(\Sigma, S, \rho, s_0, T)$ such that
\begin{itemize}
	\item $\Sigma = 2^{L}$ is the set of alphabet, where $L$ is the literal set of $\phi$;
	\item $S\subseteq 2^{\cl(\phi)}$ is the set of states;
	\item $\rho:  S \times \Sigma \to 2^S$ is the transition function, where $s_2 \in \rho(s_1, \omega) (\omega \in \Sigma)$ holds iff there is a satisfying assignment $A$ of $\xnf(s_1)^p$ such that $s_2 = X(A)$ and $\omega\models L(A)$;
	\item $s_0 = \{\phi \}$ is the initial state;
	\item $T\subseteq \rho$ is the set of accepting transitions. A transition $s_1\tran{\omega}s_2$ is in $T$ iff $\omega\models s_1$ holds. 
\end{itemize}
\end{definition}

As Definition \ref{def:ltlf2nfa} shows, a state in the \NFA is a set of subformulas of the input $\phi$. For a simple description, \textbf{we mix the usage of a state $s$ such that it can also represent a Boolean formula $\bigwedge_{\psi_i\in s}\psi_i$, i.e., the conjunction of all subformulas in $s$.} 

A run $r$ of the transition-based \NFA ${\A_{\phi}}^{tn}$ on $\eta$ is a finite state sequence $s_0 s_1\ldots$ such that $s_0$ is the initial state and $s_{i+1}\in\rho(s_i,\omega_i)$ for $0\leq i < |\eta|$. $\eta$ is accepted by ${\A_{\phi}}^{tn}$ iff there exists a run $r$ of ${\A_{\phi}}^{tn}$ on $\eta$ which ends with an accepting transition in $T$. 

Since in Definition \ref{def:ltlf2nfa}, the satisfaction of a finite trace with length one to an \ltlf formula is used to determine the accepting conditions, we introduce the following theorem to show that this process can be achieved in a much easier way than the regular satisfaction check in which the finite trace can have an arbitrary length.

\begin{theorem}\label{thm:oneSat}
Given a finite trace $\eta \in \Sigma^{+}$ with $|\eta|=1$, and an \ltlf formula $\phi$, it is true that $\eta \models \phi$ holds iff
\begin{itemize}
	\item $\phi$ is $\tt$; or 
	\item $\phi \in\eta$ when $\phi$ is a literal; or
	%\item If $\phi=\X \psi$, then return false;
	\item $\phi=\N \psi$ is a weak Next formula; or
	\item $\eta \models \phi_{2}$ holds when $\phi=\phi_{1} \U \phi_{2}$ is an Until formula; or  
	\item $\eta \models \phi_{2}$ holds when $\phi=\phi_{1} \R \phi_{2}$ is a Release formula; or
	\item $\eta \models \phi_{1}$ and $\eta \models \phi_{2}$ hold when $\phi=\phi_{1} \wedge \phi_{2}$; or
	\item $\eta \models \phi_{1}$ or $\eta \models \phi_{2}$ holds when $\phi=\phi_{1} \vee \phi_{2}$.
\end{itemize}
\end{theorem}

\begin{proof}
The theorem is self-explained which can be derived from the semantics of \ltlf.
\end{proof}

We present the following theorem to guarantee the correctness of our construction. 

\begin{theorem}\label{thm:ltlf2nfa}
Given an \ltlf formula $\phi$ and the \TNFA ${\A_{\phi}}^{tn}$ constructed by Definition \ref{def:ltlf2nfa}, a finite trace $\eta\models\phi$ holds iff $\eta$ is accepted by ${\A_{\phi}}^{tn}$. 
\end{theorem}
\begin{proof}
Assume $\eta = \omega_0\omega_1\ldots\omega_n$ ($n\geq 0$). 

($\Leftarrow$) We prove by induction over the values of $n$.
\begin{itemize}
	\item According to Definition \ref{def:ltlf2nfa}, $\eta$ is accepted by ${\A_{\phi}}^{tn}$ implies that there is a run $r=s_0s_1\ldots s_n s_{n+1}$ of ${\A_{\phi}}^{tn}$ on $\eta$ such that $\omega_n\models s_{n}$. So basically when $k = n$, it holds that $\eta_k\models s_k$;
	\item Inductively, assume $\eta_k\models s_k$ holds for $0<k\leq n$. Because $s_{k-1}\tran{\omega_{k-1}}s_k$ is a transition in the \TNFA and $\eta_k\models s_k$ holds, the set $A_{k-1} = \omega_{k-1}\cup \{\X\theta | \theta\in s_k\}$ is a satisfying assignment of $\xnf(s_{k-1})^p$, from Definition \ref{def:ltlf2nfa}. Then based on Lemma \ref{lem:propSat} and since $\eta_{k-1}=\omega_{k-1}\cdot\eta_k$ is true, we have that $\eta_{k-1}\models s_{k-1}$.
\end{itemize}
When $k=0$, we prove that $\eta (=\eta_0)\models \phi (=s_0)$ is true.

($\Rightarrow$) We first prove that $\eta\models\phi$ implies there is a run $r$ of ${\A_{\phi}}^{tn}$ on $\eta$. 
If $n = 0$, Corollary \ref{coro:propSat} shows that there is a satisfying assignment $A$ of $\xnf(\phi)^p$ such that $\eta[0]\models L(A)$. From Definition \ref{def:ltlf2nfa}, $\phi\tran{\eta[0]}X(A)$ is a transition of the \TNFA. As a result, there is a run $r=s_0(=\phi)s_1(=X(A))$ of ${\A_{\phi}}^{tn}$ on $\eta$.
 
If $n>0$, according to Lemma \ref{lem:propSat}, $\eta\models\phi$ implies there is a satisfying assignment $A$ of $\xnf(\phi)^p$ such that $\omega_0\models L(A)$ and $\eta_1\models X(A)$. From Definition \ref{def:ltlf2nfa}, $\phi (=s_0)\tran{\omega_0}X(A) (=s_1)$ is a transition in the \TNFA. Recursively applying the above process to $\eta_i\models s_i(i\geq 1)$, one can prove there is a transition $s_i\tran{\omega_i}s_{i+1}$ in the \TNFA. As a result, we have $\eta\models\phi$ implies there is a run $r$ of ${\A_{\phi}}^{tn}$ on $\eta$.  

Now we prove that the run $r$ is an accepting run. If $n=0$, the run $r$ is $s_0s_1$ such that $\eta[0]\models s_0(=\phi)$ holds. According to Definition \ref{def:ltlf2nfa}, the transition $s_0\tran{\eta[0]}s_1$ is an accepting transition. As a result, $r$ is an accepting run and $\eta$ is accepted by ${\A_{\phi}}^{tn}$.

If $n>0$, we can also prove that for the run $r=s_0s_1\ldots s_n s_{n+1}$, $s_n\tran{\omega_n}s_{n+1}$ is an accepting transition, due to the fact that $\omega_n\models s_n$. The proof is done.
\end{proof}

We discuss the upper bound of the generated \TNFA in the following theorem.

\begin{theorem}\label{thm:tnfaBound}
For an \ltlf formula $\phi$, the size of the corresponding $\TNFA$ ${\A_{\phi}}^{tn}$ generated by Definition \ref{def:ltlf2nfa} is at most $2^{|cl(\phi)|}$.
\end{theorem}
\begin{proof}
	The proof is straightforward as every state in ${\A_{\phi}}^{tn}$ is from $2^{cl(\phi)}$, according to Definition \ref{def:ltlf2nfa}.
\end{proof}

We have shown the construction from an \ltlf formula to its equivalent \TNFA. The following theorem indicates that there is a trivial way to convert a \TNFA to its equivalent \NFA. 

\begin{theorem}\label{thm:tnfa2nfa}
For a \TNFA $\A^{tn}$, there is an equivalent \NFA $\A^n$ such that $\eta$ is accepted by $\A^{tn}$ iff $\eta$ is accepted by $\A^n$, for $\eta\in \Sigma^+$. 
\end{theorem}
\begin{proof}
Assume $\A^{tn}=(\Sigma, S, \rho, s_0, T)$ and one can construct the \NFA $\A^{n} = (\Sigma, S', \rho', s_0, F)$ such that 
\begin{itemize}
	\item $S' = S\cup \{f\}$ where $f$ is a new and only accepting state in $\A^n$; 
	\item $\rho'=\rho\cup\{s_1\times\omega\to f|\ s_1\times\omega\to s_2\in T\}$;
	\item $F=\{f\}$.
\end{itemize}
We prove the equivalence of $\A^{tn}$ and $\A^n$ as follows. On one hand, every accepting run of $\A^{tn}$ which ends with an accepting transition in $T$ can also be an accepting run of $\A^n$ which ends with the accepting state $f$. On the other hand, every accepting run of $\A^{n}$ which ends with the accepting state $f$ can also be an accepting run of $\A^{tn}$ which ends with an accepting transition in $T$.
\end{proof}

