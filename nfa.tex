\section{SAT-Based Automata Construction}

In this section, we first construct the \ltlf transition system by \SAT solvers, based on which we can generate the corresponding \NFA for the given \ltlf formula. 

\subsection{\ltlf-to-\NFA construction }

To leverage \SAT solvers for the \ltlf-to-\NFA construction, we first consider \ltlf formulas as propositional formulas. The motivation comes from treating all temporal subformulas of $\phi$ as atomic propositions. 

\begin{definition}[Propositional Atoms]\label{def:pa}
For an \ltlf formula $\phi$, we define the set of propositional atoms, $\PA(\phi)$, recursively as follows.
\begin{itemize}
	\item If $\phi$ is an atom, Next, Until or Release formula, $\PA(\phi)=\{\phi\}$;
	\item If $\phi = (\neg \psi)$, $\PA(\phi)= \PA(\psi)$;
	\item If $\phi = (\phi_1 \wedge \phi_2)$ or $(\phi_1 \vee \phi_2)$, $\PA(\phi) = \PA(\phi_1) \cup \PA(\phi_2)$.
\end{itemize}
\end{definition}

Consider $\phi$ = $(a \wedge ((\neg c \wedge a)\U b) \wedge (c \wedge \X (a \vee b)))$. In this case, we have $\PA(\phi)=\{a, c, (\neg c \wedge a)\U b), (\X (a \vee b))\}$. 

\begin{definition}[Propositional Formulas]\label{def:pf}
Given an \ltlf formula $\phi$, let $\phi^{p}$ be a propositional formula over $\PA(\phi)$. 
\end{definition}

Consider $\phi$ = $(a \wedge ((\neg c \wedge a)\U b) \wedge (c \wedge \X (a \vee b)))$. From Definition \ref{def:pf}, the corresponding propositional formula $\phi^p = (a \wedge p_1) \wedge (c \wedge p_2)$, in which $p_1, p_2$ are two Boolean variables w.r.t. the truth values of $((\neg c \wedge a)\U b)$ and $(c \wedge \X (a \vee b))$. 

\begin{definition}[neXt Normal Form]\label{def:xnf}
For an \ltlf formula $\phi$, we define its Next Normal Form, i.e., $\xnf(\phi)$, recursively as follows. 
\begin{itemize}
\item If $\phi$ is a literal or $\tt$, $\xnf(\phi) = \phi \wedge \X(\tt)$; $\xnf(\ff) = \ff$;
\item If $\phi = \X(\psi)$ or $\phi = \N(\psi)$, $\xnf(\phi) = \tt \wedge \X(\psi)$;
\item If $\phi = (\phi_1 \wedge \phi_2)$, $\xnf(\phi) = \xnf(\phi_1)\wedge\xnf(\phi_2)$;
\item If $\phi = (\phi_1 \vee \phi_2), \xnf(\phi) = \xnf(\phi_1)\vee \xnf(\phi_2)$; 
\item If $\phi = (\phi_1 \U \phi_2), \xnf(\phi) = \xnf(\phi_2) \vee (\xnf(\phi_1) \wedge \X(\phi))$; 
\item If $\phi = (\phi_1 \R \phi_2), \xnf(\phi) = \xnf(\phi)\wedge (\xnf(\phi_2) \vee \N(\phi))$.
\end{itemize}
\end{definition}

By restricting the \ltlf formula $\phi$ to the XNF, we can obtain a satisfying assignment of $\phi^{p}$ with the help of a SAT solver.

\begin{definition}[Accepting State] 
For an \ltlf formula $\phi$ and $s$ being a state of the transition system $T_{\phi}$. $s$ is an accepting state iff there exists a transition $s \overset{\alpha}{\rightarrow}s$ in $T_{\phi}$ such that $\alpha \models s$ holds. 
\end{definition}
   
\begin{definition}[NFA Construction]\label{def:ltlf2nfa} 
Given an $LTL_f$ formula $\phi$ and its literal set $L$, $\Sigma = 2^{L}$, the representation of an NFA $\A$ is tuple $(\Sigma, S, \rho, s_0, F)$ such that
\begin{itemize}
	\item $\Sigma$ is the set of alphabet;
	\item $S\subseteq 2^{\cl(\phi)}$ is the set of states;
	\item $\rho:  S \times \Sigma \to 2^S$ is the transition function, where $s_2 \in \rho(s_1, \alpha) (\alpha \in \Sigma)$ holds iff $ \alpha \wedge \X(s_2) \Rightarrow \xnf(s_1)$;
	\item $s_0 = \{\phi \}$ is the initial state;
	\item $F\subseteq S$ is the set of accepting states. 
\end{itemize}
\end{definition}


\begin{lemma}\label{lem:oneSat}
For a finite trace $\eta \in \Sigma^{+}$ and \ltlf formula $\phi$, if $\eta = \omega_0$, then $\eta \models \phi$ holds iff
\begin{itemize}
	\item $\eta \models \tt$ and $\eta \not \models \ff$;
	\item If $\phi=p$ is a literal, then return true if $p \in \eta$, otherwise return false;
	\item If $\phi=\X \psi$, then return false;
	\item If $\phi=\N \psi,$ then return true;
	\item If $\phi=\phi_{1} \U \phi_{2}$ or  $\phi=\phi_{1} \R \phi_{2}$, then return  $\eta \models \phi_{2}$;
	\item If $\phi=\phi_{1} \wedge \phi_{2},$ then return $\eta \models \phi_{1}$ and $\eta \models \phi_{2}$;
	\item If $\phi=\phi_{1} \vee \phi_{2},$ then return $\eta \models \phi_{1}$ or $\eta \models \phi_{2}$.
\end{itemize}
\end{lemma}

\begin{proof}
The lemma is self-evident which can be derived from the semantics of \ltlf.
\end{proof}

Now we characterize the satisfaction relation for finite sequences whose lengths are greater than 1.

\begin{lemma}\label{lem:finiteSat}
For a finite trace $\eta = \omega_0\omega_1\ldots \omega_n \in \Sigma^{+}$ and \ltlf formula $\phi$, $\eta \models \phi$ holds iff there exists a run $r = \phi \overset{\alpha_0}{\rightarrow}\phi_1\overset{\alpha_1}{\rightarrow}\phi_2...\overset{\alpha_n}{\rightarrow}\phi_{n+1}$ in the \NFA $\A_{\phi}$ s.t. $\omega_i \models \alpha_i$ and $\eta_i \models \phi_i $ holds for every $0 \leq i \leq n$.
\end{lemma}
\begin{proof}
The Lemma above can be proved by recursively applying the first item if $n = 0$, and the second item if $n \geq 1$. 
\begin{enumerate}
	\item If $n = 0$, $\eta \models \phi$ holds iff there exists $\alpha \wedge \X\phi \in \xnf(\phi)$ or $\alpha_i \wedge \X(\tt) \in \xnf(\phi)$ s.t. $\omega_0 \models \alpha$ and $\alpha \models \phi$;
	\begin{itemize}
		\item If $\phi = p$, then $\eta \models \phi$ iff $w_0 \models$ p, and p $\models \phi$ hold. According to Definition 2, $p \wedge \X(tt) \in xnf(p)$;
		\item If $\phi=\X \psi$, according to Lemma 1, $\eta \models \X \psi$ is always false. Similiarly, If $\phi=\N \psi$, $\eta \models \N \psi$ is always true;
		\item If $\phi=\phi_{1} \U \phi_{2}$, according to Lemma 1, $\eta \models \phi$ iff $\eta \models \phi_{2}$. $xnf(\phi_{2})=\phi_{2} \wedge \X(tt)$, then  $w_0 \models \phi_{2}$, and $\phi_{2} \models \phi_{2}$, thus the proof is done. The proof for the case $\phi=\phi_{1} \R \phi_{2}$ is similar;
		\item If $\phi=\phi_{1} \wedge \phi_{2}$, according to Definition 3, there exists $\beta_i \wedge \X\psi_i \in xnf(\phi)(i = 1,2).$ Let $\alpha = \beta_1 \wedge \beta_2$ and $\psi = \phi_1 \wedge \phi_2$, then $w_0 \models \alpha$ and $\alpha \models \phi$ hold. Similarly, the proof for the case when $\phi = \phi_1 \vee \phi_2$ is the same.
	\end{itemize}
	\item  If $n \geq 1$, $\eta \models \phi$ holds iff there exists $\alpha_i \wedge \X\phi_i \in \xnf(\phi)$ s.t. $\omega_0 \models \alpha_i$ and $\eta_1 \models \phi_i$.
	\begin{itemize}
		\item If $\phi_i = p$, then $\eta \models \phi$ iff $w_0 \models \phi$, and $\eta_1 \models tt$, according to Definition 2, $xnf(p)=p \wedge \X(tt)$, then proof is done.\\
 		\item If $\phi_i=\X \psi$ or $\eta \models \N \psi$, then $\eta \models \phi$ iff $w_0 \models tt$ and $\eta_1 \models \psi$. According to Lemma 1, $tt \wedge \X\phi_2 \in xnf(\phi)$.\\
		\item If $\phi=\phi_1 \wedge \phi_2$, then $\eta \models \phi_1 \U \phi_2$ iff $\eta \models \phi_1$ and $\eta \models \phi_2$, and iff there exists $\beta_i \wedge \X\psi_i \in xnf(\phi_i)(i=1,2)$ s.t. $w_0 \models \beta_1 \wedge \beta_2$ and $\eta_1 \models \psi_1 and \psi_2$. According to Definition ,$(\beta_1 \wedge \beta_2)\wedge \X(\psi_1 and \psi_2)$. The proof for the case when $\phi_i=\psi_{1} \vee \psi_{2}$ is similar. \\
 		\item If $\phi_i=\phi_{1} \U \phi_{2}$, then $\eta \models \phi$ iff $\eta \models \phi_2$ or $\eta \models (\phi_1 \wedge \X\phi)$. If $\eta \models \phi_2$ holds, then $\eta \models \phi$ iff there exists $\alpha_i \wedge \X\phi_i \in xnf(\phi_2)$ s.t. $w_0 \models \alpha_i$ and $\eta_1 \models \phi_i$. According to Definition , there exists $\alpha_i \wedge \X\phi_i \in xnf(\phi_2)$. On the other hand, if $\eta \models (\phi_1 \wedge \X\phi)$ holds, the proof for the case when the operator is $\wedge$ is done. So $\eta \models \phi$ iff there exists $\alpha_i \wedge \phi_i \in xnf(\phi_2)$ s.t. $w_0 \models \alpha_i$ and $\eta_1 \models \phi_i$. The proof the case when $\phi=\phi_1 \R \phi_2$ is similar.
	\end{itemize}
\end{enumerate}

\end{proof}

Lemma \ref{lem:finiteSat} denotes that, to check whether a finite trace $\eta = \omega_0 \omega_1\ldots \omega_n$ satisfies an \ltlf formula $\phi$, iff we can find a trace of an \NFA $\A_\phi$ s.t. $\eta$ can finally reach an accepting state. Now we show the main theorem of this paper.

\begin{theorem}[Main Theorem]  
For a finite trace  $\eta = \omega_0\omega_1\ldots \omega_n $ and \ltlf formula $\phi$, $\eta \models \phi$ iff $\eta$ can be accepted by the \NFA $\A_\phi$.
\end{theorem}



