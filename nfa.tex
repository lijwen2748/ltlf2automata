\section{SAT-Based Automata Construction}
In this section, we construct the $LTL_f$ transition system by SAT solvers. Before defining the transition system, we need to introduce some definitions.\\
\subsection{$LTL_f$ Transition System}
First, we need to consider $LTL_f$ formulas as propositional formulas. And to this end, all temporal subformulas of $\phi$ are treated as atomic propositions. \\
{\bf Definition 1} (Propositional Atoms). {\it For an $LTL_f$ formula $\phi$, we define $PA(\phi)$ as follows:\\
1) If $\phi$ is an atom, Next, Until or Release formula, $PA(\phi)=\{\phi\}$;\\
2) If $\phi = (\neg \psi)$, then $PA(\phi)= PA(\psi)$;\\
3) If $\phi = (\phi_1 \wedge \phi_2)$ or $(\phi_1 \vee \phi_2)$, then $PA(\phi) = PA(\phi_1) \cup PA(\phi_2)$.
}     \\
Consider $\phi$ = $(a \wedge ((\neg c \wedge a)U b) \wedge \neg(c \wedge X (a \vee b)))$. In this case, we have $PA(\phi)=\{a, c, (\neg c \wedge a)U b), (X (a \vee b))\}$. 
{\bf Definition 1} (Next Normal Form). {\it For an $LTL_f$ formula $\phi$, $\phi$ is in Next Normal Form(XNF) if every subformula $X \phi$ is of the form $X X \phi^{\prime}, X a$ or $X \neg a$ for some $a \in \Sigma$ which also means that there are no Until or Release subformulas of $\phi$ in $PA(\phi)$. }     \\
In [], Li et al. have denoted the normal form of an $LTL_f$ formula $\phi$ as $NF(\phi)$.\\
{\bf Definition 2} (Normal Form). {\it The normal form of an $LTL_f$ formula $\phi$ is defined as follows: \\
$\bullet$ $NF(\phi) = \{\phi \wedge X(tt)\}$ if $\phi \not \equiv ff$ is a propositional formula. If $\phi \equiv ff$, we define   $NF(ff) = \phi$;\\
$\bullet$ $NF(X\phi/X_{w}\phi) = \{tt \wedge X(\psi)| \psi \in DF(\phi)\}$; \\
$\bullet$ $NF(\phi_1 U \phi_2) = NF(\phi_2) U NF(\phi_1 \wedge X(\phi_1 U \phi_2))$; \\
$\bullet$  $NF(\phi_1 R \phi_2) = NF(\phi_1 \wedge \phi_2) U NF(\phi_2 \wedge X(\phi_1 R \phi_2)$; \\
$\bullet$ $NF(\phi_1 \vee \phi_2) = NF(\phi_1) U NF(\phi_2)$; \\
$\bullet$ $NF(\phi_1 \wedge \phi_2) = \{(\alpha_1 \wedge \alpha_2) \wedge X(\psi_1 \wedge \psi_2)| \forall i = 1,2. \alpha_i \wedge X(\psi_i)\}$; \\

}
{\bf Definition 3} ($LTL_f$ Transition System). {\it Given an $LTL_f$ formula $\phi$ and its literal set L, $\Sigma = 2^{L}$. We define the corresponding transition system $T_{\phi}$ as a turple $(s_0,S, T)$ where: \\
1) $s_0 = \{\phi\}$ is the initial state; \\
2) S is the set of states;\\
3) $T : S \times \Sigma \to S$ is the transition relation that is defined by: $s_1\overset{\alpha}{\rightarrow} s_2$ iff there exists $\alpha \wedge X(s_2) \in NF(s_1)$; \\
A run of $T_{\phi}$ on a finite trace $\eta = w_0w_1...w_n \in \Sigma^{*}$ is a run $s_0 \overset{\alpha_0}{\rightarrow} s_1 \overset{\alpha_1}{\rightarrow} ... s_n \overset{\alpha_n}{\rightarrow} s_{n+1} $ s.t. $s_0 = \{\phi\}$,  $\alpha_i \wedge X(s_{i+1}) \in NF(s_i)$ holds for all $0 \leq i \leq n$.
\\}
{\bf Definition 4} (Acceptable State). {\it Given an $LTL_f$ formula $\phi$. Suppose that $s_1$ is a state of a NFA $T_{\phi}$. We say  $s_1$ in $T_{\phi}$ is an acceptable state iff there exists a transition $ s_1 \overset{\alpha}{\rightarrow}s_1$, and moreover $CF(\alpha) \models s_1$. }   \\   
\section{Approach Overview}
Given an $LTL_f$ formula $\phi$, there is a NFA $A_{\phi}$ which accepts exactly the same language as $\phi$. To construct the nondeterministic finite automaton (NFA) for an $ LTL_f$ formula $\phi$, we leverage SAT solvers to find all one-transition next states of the current state, then check the acceptable states. 

\subsection{LTL to NFA}
Given an  $ LTL_f$ formula, We define the construction of transition system as follows. \\
  \begin{algorithm}[H]
    \SetAlgoNoLine
 \SetKwInOut{Input}{\textbf{Input}}\SetKwInOut{Output}{\textbf{Output}}
 \Input{ An $ LTL_f$ formula $\phi$\\}
    \Output{ transition system S\\}
    \BlankLine
 $to\_process:=\{\phi\}, S:=\{\phi\}$, \\
\textbf{set} existing to mark visited states.\\
\textbf{while} $to\_process$ is not empty \textbf{do} \\
\ cur =  $to\_process$.front;  \\
\quad  \textbf{while} true \textbf{do} \\
\qquad   \textbf{if} cur is not satisfiable  \textbf{then} \{ \\
\qquad \quad $to\_process$.pop; \\
\qquad \quad break;     \}\\
 \qquad  Let t=(label,next) a transition edge from current state(cur) \\
\qquad\textbf{if} t.next is not in existing \textbf{then} \{ \\
\qquad \quad $to\_process$.push(t.next); \\
\qquad \quad existing.push(t.next); \\
\qquad  \quad S.push(t.next)  \}\\
\qquad  add transition edge t to the current state(cur)
 \caption{Construction of Transition System}
\end{algorithm}

The algorithm constructs the transition system on-the-fly. The initial state is set to be $\{\phi\}$, which is the input formula. 
\subsection{Checking Acceptable States}
In this section, we present our algorithm for checking acceptable states of $LTL_f$ formulas. The lemma  was proposed by (Jianwen Li ) at [].  \\

{\bf Lemma 1.}  {\it For a finite trace $\eta \in \Sigma^{*}$ and $LTL_f$ formula $\phi$, if $\eta = w_0$, then $\eta \models \phi$ holds iff: \\ 
$\bullet$ $\eta \models tt$ and $\eta \not \models $ ff;\\
$\bullet$ If $\phi=p$ is a literal, then return true if $p \in \eta$, otherwise return false;\\
$\bullet$ If $\phi=X \psi$, then return false;\\
$\bullet$ If $\phi=X_{w} \psi,$ then return true;\\
$\bullet$ If $\phi=\phi_{1} U \phi_{2}$ or  $\phi=\phi_{1} R \phi_{2}$, then return  $\eta \models \phi_{2}$; \\
$\bullet$ If $\phi=\phi_{1} \wedge \phi_{2},$ then return $\eta \models \phi_{1}$ and $\eta \models \phi_{2}$; \\
$\bullet$ If $\phi=\phi_{1} \vee \phi_{2},$ then return $\eta \models \phi_{1}$ or $\eta \models \phi_{2}$.\\}

\begin{proof}
The lemma is self-evident,and we can derive it from the semantics of $LTL_f$.
\end{proof}

Now we characterize the satisfaction relation for finite sequences: \\
{\bf Lemma 2.}  {\it For a finite trace $\eta = w_0w_1...w_n \in \Sigma^{*}$ and $LTL_f$ formula $\phi$.
1. If n = 0, then $\eta \models \phi$ iff there exists $ \alpha \wedge X\phi \in NF(\phi)$ s.t. $w_0 \models \alpha, CF(\alpha)\models \phi$.
}
\begin{proof}
$\bullet$ If $\phi = p$, according to Definition 2, NF(p)=p $\wedge$ X(tt), then $w_0 \models$ p, and CF(p) $\models p$ s.t. $\eta \models \phi$ .\\
$\bullet$ If $\phi=X \psi$, according to Lemma 1, $\eta \models X \psi$ is always false. Similiarly, If $\phi=X_{w} \psi$, $\eta \models X_{w} \psi$ is always true.\\
$\bullet$ If $\phi=\phi_{1} U \phi_{2}$, according to Lemma 1, $\eta \models \phi_{1} U \phi_{2}$ iff $\eta \models \phi_{2}$.$NF(\phi_{2})=\phi_{2} \wedge X(tt)$, then  $w_0 \models \phi_{2}$, and $CF(\phi_{2}) \models \phi_{2}$ s.t. $\eta \models \phi$ . The proof for the case $\phi=\phi_{1} R \phi_{2}$ is similar. \\




\end{proof}
        
To check whether a state is an acceptable state, we can discuss it in the following two steps: \\
1. If $s_1 $= tt, return true; \\
2. If there exists a transition  $ s_1 \overset{\alpha}{\rightarrow}s_1$ and  the properties on the edge satisfy the current state, then return true. \\

\IncMargin{1em}
 \begin{algorithm}[H]
    \SetAlgoNoLine
 \SetKwInOut{Input}{\textbf{Input}}\SetKwInOut{Output}{\textbf{Output}}
 \Input{ An $ LTL_f$ formula $\phi$\\}
    \Output{  acceptable states Accept}
    \BlankLine
\textbf{for} i $\leftarrow$  0 to S.size  \\
\qquad cur = S[i].property;  \\
\qquad  \textbf{if} cur is 'true'   \textbf{then}\{      \\
\qquad \quad Accept.push\_back(i);     \\
\qquad \quad  continue;  \}     \\
\qquad  $ \textbf{for}$ j $\leftarrow$  0 to S[i].trans\_set.size \\
\qquad \quad state\_next = S[i].trans\_set[j].next;  \\
\qquad \quad \textbf{if} state\_next == i and S[i].trans\_set[j].label satisfies current state \textbf{then} \{  \\
\qquad \qquad Accept.push\_back(i);  \\
\qquad \qquad \textbf{break}; \} \\
 \caption{Checking Acceptable States}
\end{algorithm}
