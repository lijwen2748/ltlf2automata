\section{SAT-Based Automata Construction}
In this section, we construct the $LTL_f$ transition system by SAT solvers. Before defining the transition system, we need to introduce some definitions.\\
\subsection{$LTL_f$ Transition System}
First, we need to consider $LTL_f$ formulas as propositional formulas. And to this end, all temporal subformulas of $\phi$ are treated as atomic propositions. \\
{\bf Definition 1} (Propositional Atoms). {\it For an $LTL_f$ formula $\phi$, we define $PA(\phi)$ as follows:\\
1) If $\phi$ is an atom, Next, Until or Release formula, $PA(\phi)=\{\phi\}$;\\
2) If $\phi = (\neg \psi)$, then $PA(\phi)= PA(\psi)$;\\
3) If $\phi = (\phi_1 \wedge \phi_2)$ or $(\phi_1 \vee \phi_2)$, then $PA(\phi) = PA(\phi_1) \cup PA(\phi_2)$.
}     \\
Consider $\phi$ = $(a \wedge ((\neg c \wedge a)\U b) \wedge (c \wedge \X (a \vee b)))$. In this case, we have $PA(\phi)=\{a, c, (\neg c \wedge a)\U b), (\X (a \vee b))\}$. 
{\bf Definition 2} (Propositional Formulas). {\it Given an $LTL_f$ formula $\phi$, let $\phi^{p}$ be a propositional formula over $PA(\phi)$. \\}
Consider  $\phi$ = $(a \wedge ((\neg c \wedge a)\U b) \wedge (c \wedge \X (a \vee b)))$. From definition 2, the corresponding propositional formula $\phi^p = (a \wedge p_1) \wedge (c \wedge p_2)$, in which $p_1, p_2$ are two boolean variables denoting the truth values of $((\neg c \wedge a)\U b)$ and $(c \wedge \X (a \vee b))$. \\
{\bf Definition 3} (Next Normal Form). {\it For an $LTL_f$ formula $\phi$, $\phi$ is in Next Normal Form(XNF) if every subformula $\X \phi$ is of the form $\X \X \phi^{\prime}, \X a$ or $\X \neg a$ for some $a \in \Sigma$. The Next Normal Form $LTL_f$ formula $\phi$ is defined as follows:\\ 
1) If $\phi$ is a literal l, $xnf(\phi) = \phi \wedge \X(tt)$;\\
2)  $xnf(tt) =  tt$, $xnf(ff) =  \varnothing$;\\
3) If $\phi = \X(\psi)$, $xnf(\phi) = tt \wedge \X(\psi) $;\\
4) If $\phi = (\phi_1 \wedge \phi_2)$, $xnf(\phi) = \{\alpha_1 \wedge \alpha_2)\wedge \X(\psi_1 \wedge \psi_2)| \forall i = 1,2.$ $\alpha_i \wedge \X(\psi_i)\in xnf(\phi_i)$; \\
5) If $\phi = (\phi_1 \vee \phi_2), xnf(\phi) = xnf(\phi_1)\vee xnf(\phi_2)$; \\
6) If $\phi = (\phi_1 \U \phi_2), xnf(\phi) = xnf(\phi_2) \vee (xnf(\phi_1) \wedge \X(\phi))$; \\
7) If $\phi = (\phi_1 \R \phi_2), xnf(\phi) = xnf(\phi)\vee (xnf(\phi_2) \wedge \X(\phi))$; \\
}    
By restricting $LTL_f$ formulas $\phi$ to XNF, we can obtain a satisfying assignment of $\phi^{p}$, leveraging a SAT solver.\\
{\bf Definition 4} (Accepting States). {\it Given an $LTL_f$ formula $\phi$. Suppose that $s_1$ is a state of a transition system $T_{\phi}$. We say  $s_1$ is an accepting state iff there exists a transition $ s_1 \overset{\alpha}{\rightarrow}s_1$, and moreover $\alpha \models s_1$. }   \\   
{\bf Definition 5} (Non-deterministic Finite Automaton). {\it An non-deterministic finite automaton is referred to by its acronym: NFA. Given an $LTL_f$ formula $\phi$ and its literal set L, $\Sigma = 2^{L}$. The representation of an NFA is a listing of five components: $A = (s_0,S, T, \Sigma, F)$ where:
1) $s_0 = \{\phi \}$ is the initial state.\\
2) S is a finite set of states.\\
3) $T:  S \times \Sigma \to 2^S$ is the transition function, where $s_2 \in T(s_1, \alpha) (\alpha \in \Sigma)$ holds iff $ \alpha \wedge \X(s_2) \in xnf(s_1)$.\\
4) $\Sigma$ is a finite set of input symbols.\\
5) F, a subset of S, is the set of accepting states.\\ }
Intuitively, $T(s, a)$ is the set of states that A can move into when it is in state s and it reads the symbol a.  \\
A run r of A on a finite word $w = a_0, ..., a_{n-1} \in \Sigma^{*}$ is a sequence $s_0, ..., s_n $of n+1 states in S such that $s_0 \in S_0$, and $s_{i+1} \in T(s_i, a_i)$ for $0 \leq i < n$. Notably, a nondeterministic automaton can have many runs on a given input word. A run r is accepting if $s_n \in F$.   \\
{\bf Lemma 1.}  {\it For a finite trace $\eta \in \Sigma^{*}$ and $LTL_f$ formula $\phi$, if $\eta = w_0$, then $\eta \models \phi$ holds iff: \\ 
$\bullet$ $\eta \models tt$ and $\eta \not \models $ ff;\\
$\bullet$ If $\phi=p$ is a literal, then return true if $p \in \eta$, otherwise return false;\\
$\bullet$ If $\phi=\X \psi$, then return false;\\
$\bullet$ If $\phi=\N \psi,$ then return true;\\
$\bullet$ If $\phi=\phi_{1} \U \phi_{2}$ or  $\phi=\phi_{1} \R \phi_{2}$, then return  $\eta \models \phi_{2}$; \\
$\bullet$ If $\phi=\phi_{1} \wedge \phi_{2},$ then return $\eta \models \phi_{1}$ and $\eta \models \phi_{2}$; \\
$\bullet$ If $\phi=\phi_{1} \vee \phi_{2},$ then return $\eta \models \phi_{1}$ or $\eta \models \phi_{2}$.}

\begin{proof}
The lemma is self-evident which can be derived from the semantics of $LTL_f$.
\end{proof}

Now we characterize the satisfaction relation for finite sequences: \\
{\bf Lemma 2.}  {\it For a finite trace $\eta = w_0w_1...w_n \in \Sigma^{*}$ and $LTL_f$ formula $\phi$. $\eta \models \phi$ iff there exists a run $r = \phi \overset{\alpha_0}{\rightarrow}\phi_1\overset{\alpha_1}{\rightarrow}\phi_2...\overset{\alpha_n}{\rightarrow}\phi_{n+1}$ in NFA $A_{\phi}$ s.t. every $0 \leq i \leq n$ it holds that $w_i \models \alpha_i$ and $\eta_i \models \phi_i $.}
\begin{proof}
The Lemma above can be proved by recursively applying the first item if $n = 0$, and the second item if $n \geq 1$. \\
1) If n = 0, then $\eta \models \phi$ iff there exists $ \alpha \wedge \X\phi \in xnf(\phi)$ or $\alpha_i \wedge \X(tt) \in xnf(\phi)$ s.t. $w_0 \models \alpha, \alpha \models \phi$.\\
2) If $n \geq 1$, then  $\eta \models \phi$ iff there exists  $ \alpha_i \wedge \X\phi_i \in xnf(\phi)$ s.t. $w_0 \models \alpha_i$ and $\eta_1 \models \phi_i $.

1) $\bullet$ If $\phi = p$, then $\eta \models \phi$ iff $w_0 \models$ p, and p $\models \phi$ hold. According to Definition 2, $p \wedge \X(tt) \in xnf(p)$.\\
$\bullet$ If $\phi=\X \psi$, according to Lemma 1, $\eta \models \X \psi$ is always false. Similiarly, If $\phi=\N \psi$, $\eta \models \N \psi$ is always true.\\
$\bullet$ If $\phi=\phi_{1} \U \phi_{2}$, according to Lemma 1, $\eta \models \phi$ iff $\eta \models \phi_{2}$. $xnf(\phi_{2})=\phi_{2} \wedge \X(tt)$, then  $w_0 \models \phi_{2}$, and $\phi_{2} \models \phi_{2}$, thus the proof is done. The proof for the case $\phi=\phi_{1} \R \phi_{2}$ is similar. \\
$\bullet$ If $\phi=\phi_{1} \wedge \phi_{2}$, according to Definition 3, there exists $\beta_i \wedge \X\psi_i \in xnf(\phi)(i = 1,2).$ Let $\alpha = \beta_1 \wedge \beta_2$ and $\psi = \phi_1 \wedge \phi_2$, then $w_0 \models \alpha$ and $\alpha \models \phi$ hold. Similarly, the proof for the case when $\phi = \phi_1 \vee \phi_2$ is the same. \\
2) The second item can be proved by structural induction over $\phi$.\\
 $\bullet$ If $\phi_i = p$, then $\eta \models \phi$ iff $w_0 \models \phi$, and $\eta_1 \models tt$, according to Definition 2, $xnf(p)=p \wedge \X(tt)$, then proof is done.\\
 $\bullet$ If $\phi_i=\X \psi$ or $\eta \models \N \psi$, then $\eta \models \phi$ iff $w_0 \models tt$ and $\eta_1 \models \psi$. According to Lemma 1, $tt \wedge \X\phi_2 \in xnf(\phi)$.\\
$\bullet$ If $\phi=\phi_1 \wedge \phi_2$, then $\eta \models \phi_1 \U \phi_2$ iff $\eta \models \phi_1$ and $\eta \models \phi_2$, and iff there exists $\beta_i \wedge \X\psi_i \in xnf(\phi_i)(i=1,2)$ s.t. $w_0 \models \beta_1 \wedge \beta_2$ and $\eta_1 \models \psi_1 and \psi_2$. According to Definition ,$(\beta_1 \wedge \beta_2)\wedge \X(\psi_1 and \psi_2)$. The proof for the case when $\phi_i=\psi_{1} \vee \psi_{2}$ is similar. \\
 $\bullet$ If $\phi_i=\phi_{1} \U \phi_{2}$, then $\eta \models \phi$ iff $\eta \models \phi_2$ or $\eta \models (\phi_1 \wedge \X\phi)$. If $\eta \models \phi_2$ holds, then $\eta \models \phi$ iff there exists $\alpha_i \wedge \X\phi_i \in xnf(\phi_2)$ s.t. $w_0 \models \alpha_i$ and $\eta_1 \models \phi_i$. According to Definition , there exists $\alpha_i \wedge \X\phi_i \in xnf(\phi_2)$. On the other hand, if $\eta \models (\phi_1 \wedge \X\phi)$ holds, the proof for the case when the operator is $\wedge$ is done. So $\eta \models \phi$ iff there exists $\alpha_i \wedge \phi_i \in xnf(\phi_2)$ s.t. $w_0 \models \alpha_i$ and $\eta_1 \models \phi_i$. The proof the case when $\phi=\phi_1 \R \phi_2$ is similar.
\end{proof}
Lemma 2 demotes that, to check whether a finite trace $\eta = w_0 w_1 ... w_n$ satisfies an $LTL_f$ formula $\phi$, iff we can find a trace of an NFA $A_\phi$ s.t. $\eta$ can finally reach an accepting state. Now we show the main theorem of this paper.\\
{\bf Theorem 1.}  {\it For a finite trace  $\eta = w_0w_1...w_n $ and $LTL_f$ formula $\phi$, $\eta \models \phi$ iff $\eta$ can be accepted by the NFA $A_\phi$.}
\begin{proof} $(\Leftarrow)$ If $\eta$ can be accepted by the NFA $A_{\phi}$, then there exists a finite trace $\eta$ s.t. the corresponding run r of NFA $A_{\phi}$ on $\eta$ ends with  an accepting state $\phi_n$. According to Lemma 2, $\eta \models \phi$. \\
$(\Rightarrow)$ Since $\eta$ is a finite trace, $\eta \models \phi$. According to Lemma 2, there is a run r= $\phi_0 \to \phi_1 \to ...\phi_n$ where $\phi_n$ is an accepting state. Then $\eta \models \phi$(according to Definition 6).
\end{proof}

